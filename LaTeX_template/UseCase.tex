\documentclass[12pt, a4paper]{article}

% --- Preamble ---
\usepackage{amsmath}
\usepackage{amssymb}
\usepackage{graphicx}
\usepackage[margin=1in]{geometry}
\usepackage{fancyhdr}
\usepackage[colorlinks=true, allcolors=blue]{hyperref}
\usepackage[italian]{babel}
\usepackage{float}
\usepackage{url}
\usepackage{subfiles}
\usepackage{booktabs}

%\newcommand{\nuovaOff}{Creazione Offerta}

% --- Main Document ---
\begin{document}
\subsection{Individuare Casi d’uso:}
Di seguito riporto le tabelle dei casi d'uso attivi sull'applicativo divisi per
tipologia utente, escludendo le azioni di pura consultazione.\@
\begin{table}[H]
      \scalebox{0.9}{
            \begin{tabular}{l}
                  \toprule
                  \textbf{Azione}                         \\
                  \midrule
                  Creazione/Modifica Offerta Commessa     \\
                  \midrule
                  Creazione/Modifica Richiesta Assistenza \\
                  \midrule
                  Creazione/Modifica Clienti              \\
                  \midrule
                  Creazione/Modifica riferimenti Esterni  \\
                  \bottomrule
            \end{tabular}}

      \caption{Casi d'uso Commerciali}\label{tab:TableCommEs}
\end{table}

\begin{table}[H]
      \scalebox{0.9}{
            \begin{tabular}{l}
                  \toprule
                  \textbf{Azione}                         \\
                  \midrule
                  Creazione/Modifica Offerta Commessa     \\
                  \midrule
                  Creazione/Modifica Richiesta Assistenza \\
                  \midrule
                  Creazione clienti                       \\
                  \midrule
                  Aggiungere Autori                       \\
                  \midrule
                  Creazione Agente                        \\
                  \midrule
                  Creazione Nazione                       \\
                  \midrule
                  Creazione Zone                          \\
                  \midrule
                  Aggiunta/Revisione Prodotti             \\
                  \midrule
                  Creazione riferimenti Esterni           \\
                  \bottomrule
            \end{tabular}}

      \caption{Casi d'uso Commerciali Interni}\label{tab:TableComm}
\end{table}

\begin{table}[H]
      \scalebox{0.9}{
            \begin{tabular}{l}
                  \toprule
                  \textbf{Azione}                      \\
                  \midrule
                  Esportazione Contratti al Gestionale \\
                  \bottomrule
            \end{tabular}}

      \caption{Casi d'uso Amministrazione}\label{tab:TableAmm}
\end{table}

L'admin oltre alle azione precedenti può inoltre tramite apposite maschere:
\begin{table}[H]
      \scalebox{0.9}{
            \begin{tabular}{l}
                  \toprule
                  \textbf{Azione}                       \\
                  \midrule
                  Creazione/Modifica Corrieri           \\
                  \midrule
                  Creazione/Modifica Provvigioni        \\
                  \midrule
                  Creazione/Modifica prezzi prodotti    \\
                  \midrule
                  Creazione/Modifica Banche di Appoggio \\
                  \midrule
                  Creazione Lingue                      \\
                  \bottomrule
            \end{tabular}}

      \caption{Casi d'uso Admin}\label{tab:Admin}
\end{table}
\paragraph{Nota:}
Tutte le ulteriori azioni, come creazione query, modifica dell'applicativo e
aggiunta valori esterni dalle maschere preimpostate, rimangono a carico dello
sviluppatore interno.

\begin{figure}[H]\label{fig:Schema Casi D Uso}
      \centering
      \includegraphics[width=1\textwidth]{../image/schemi/CasiDUso.png}
      \caption{Schema Casi d'Uso}
\end{figure}

\end{document}