\documentclass[12pt, a4paper]{article}

% --- Preamble ---
\usepackage{amsmath}
\usepackage{amssymb}
\usepackage{graphicx}
\usepackage[margin=1in]{geometry}
\usepackage{fancyhdr}
\usepackage[colorlinks=true, allcolors=blue]{hyperref}
\usepackage[italian]{babel}
\usepackage{float}
\usepackage{cite}
\usepackage{url}
\usepackage[utf8]{inputenc}
\usepackage{booktabs}
\usepackage{hyperref}

% --- Main Document ---
\begin{document}

Dopo l'analisi si è concluso che il programma, per quanto valido nel suo lungo
periodo di utilizzo, non è la soluzione per un'azienda che ha come obiettivo la
crescita qualitativa e quantitativa. Il programma presenta problemi già
descritti nelle sezioni precedenti, come problemi di sicurezza e affidabilità,
oltre alla impossibilità di lavorare concorrentemente sul DB.\@ La manutenzione
e correzione dello stesso non è una via percorribile; la scelta migliore è
ripensare il tutto anche per unificare i codici in uso dai commerciali quelli
dei reparti tecnici, riducendo i margini di errore successivi.\@

\subsection{Proposte di Implementazione}
\begin{itemize}
      \item Riprogettare il database spostandolo su prodotti più moderni come ad esempio
            SQL Server, con la possibilità di mantenere attiva la funzionalità di creare
            offerte offline.
      \item Ripensare tutta l'interfaccia front-end in un'ottica più vicina ai canoni della
            User Experience, tenendo conto, dove possibile, anche dell'accessibilità.
      \item Inserire un configuratore di prodotto in modo da ricreare le macchine vendute
            con una corrispondenza 1:1 con quello che necessitano i tecnici in fase
            progettuale post-vendita. L'azienda prevede come beneficio di tale strumento
            una maggiore velocità in fase di vendita, ma soprattutto far fronte a tutte le
            discrepanze gestionali con i reparti successivi al commerciale, snellendo così
            tutti i processi.
      \item Includere un configuratore 3D, conseguentemente al punto precedente, che
            permetta di generare modelli preliminari delle macchine vendute da mostrare in
            fase di vendita. La realizzazione di questo configuratore si potrebbe ottenere
            appoggiandosi al programma di disegno tecnico Solid Work già in
            uso~\cite{solidworks2025}.
      \item A prescindere dall'infrastruttura, serve tenere conto del debito tecnico dei
            programmi utilizzati per rendere il software aggiornabile e manutenibile nel
            lungo periodo.
\end{itemize}

\end{document}