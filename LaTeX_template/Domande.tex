\documentclass[12pt, a4paper]{article}
% --- Preamble ---
\usepackage{amsmath}
\usepackage{amssymb}
\usepackage{graphicx}
\usepackage[margin=1in]{geometry}
\usepackage{fancyhdr}
\usepackage[colorlinks=true, allcolors=blue]{hyperref}
\usepackage[italian]{babel}
\usepackage{float}
\usepackage{cite}
\usepackage{url}
\usepackage[utf8]{inputenc}
\usepackage{booktabs}
\usepackage{hyperref}
\usepackage{subfiles}

% --- Main Document ---
\begin{document}

\subsection{Interviste all'User ed Admin del Database}
\subsubsection{Intervista all' Utente}

\paragraph{Problematiche riscontrate:}
Le problematiche si dividono in due macro argomenti: Sicurezza e Affidabilità.

\subparagraph{Sicurezza:}
\begin{itemize}
      \item Possibilità di manomissione offerte altrui senza permessi e tracciamento della
            modifica.\@
      \item Possibilità per l'utente standard di manomettere manualmente le tabelle e
            aggiungere utenti con relativa password.\@
      \item L'anagrafica del Cliente deve essere scritta due volte sia per commessa che
            assistenza; ciò comporta l'obbligo di creare manualmente due volte lo stesso
            cliente.\@ \end{itemize}

\subparagraph{Affidabilità:}
\begin{itemize}
      \item Il sistema si blocca facilmente se vengono eseguite più operazioni alla volta;
            sono necessari riavvii del sistema per risolverlo. Comuni nella schermata di
            Creazione Offerta (Figura:~\ref{fig:SchermataOfferta}) \@
      \item Se presente la concorrenza di due persone su una stessa commessa il sistema si
            blocca irrimediabilmente per tutti. Necessaria una uscita di tutti gli user dal
            DB.\@
      \item Non sono gestite le revisioni delle anagrafiche componenti e clienti, in caso
            di modifica non c'è la possibilità di vedere cosa venduto con esattezza in una
            certa data. Rimediato tenendo annualmente copie del DB.\@
      \item Non presenti controlli e blocchi nella creazione di clienti ed offerte,
            possibilità di creare copie dello stesso.\@
\end{itemize}

\begin{figure}[H]\label{fig:SchermataOfferta}
      \centering
      \includegraphics[width=0.8\textwidth]{../image/SchermataOfferta.png}
      \caption{Schermata creazione Offerta}
\end{figure}

\subsubsection{Intervista Progettista}
\begin{itemize}
      \item \textbf{Anno di progettazione:} Il programma è stato impostato in una prima stesura nel 1995 e ultimato nel 1998.
            La progettazione era partita da una base di contratti  svolti in Word.\@
      \item \textbf{Tecnologie del tempo e motivazione utilizzo Access:} Al tempo era la
            soluzione più comoda avendo a disposizione sia una interfaccia front-end che
            backend.\@
            Era disponibile sul mercato anche Oracle, ma non si adattava alle
            esigenze aziendali.\@
      \item \textbf{Logica costruzione delle Tabelle:} Le tabelle erano inizialmente
            progettate con tutte le cardinalità e i criteri di gestione tipici di un
            database RDBMS.\@
            Nel tempo, però, molte caratteristiche strutturali sono
            state perse a favore della flessibilità. Un esempio sono i vincoli
            referenziali e le relazioni tra le tabelle, che oggi vengono ricostruite
            solo mediante l'esecuzione di query.\@
      \item \textbf{Ampliamenti avvenuti nel corso del tempo:} Il database è stato
            modificato per far fronte sia a cambiamenti geopolitici, come
            l'introduzione dell'euro e la gestione di paesi UE ed extra-UE, sia a
            esigenze interne come la separazione dei ruoli tra venditori e riferimenti esterni
            delle vendite.\@
            Inizialmente i ruoli coincidevano, ma con l'aumento del
            mercato è stato necessario suddividerli. Una modifica sostanziale è
            stata lo spostamento di alcune tabelle dal back-end al front-end per
            permettere l'importazione di valori da copie locali del database.\@
      \item \textbf{Tecnologie utilizzate:}
            \begin{itemize}
                  \item \textbf{Access:} Utilizzato come front-end e back-end tramite il file \.mdb nativo di
                        accesso dove raccoglie l'intero DB.\@
                  \item \textbf{VBA:}\@ Utilizzato per richiamare le query all'interno delle maschere, scegliere
                        DB esterni da importare.\@
                  \item \textbf{Excel:} Utilizzato come tabella di frontiera per importare le offerte nel
                        programma gestione aziendale (Ad-Hoc).\@
            \end{itemize}
      \item \textbf{Prove di modernizzazione:} Nel corso degli anni si è tentato di
            modernizzare il programma mantenendo Access come front-end e SQL Server
            come back-end, già in uso per altri applicativi aziendali. La migrazione tuttavia non
            è stato possibile attuarla completamente a causa dei vincoli
            di portabilità in aree senza connessione dati, funzione fondamentale per
            la vendita in zone non coperte da Internet.
      \item \textbf{Problemi attuali:}
            \begin{itemize}
                  \item I codici creati nelle offerte non corrispondono ai codici del gestionale
                        utilizzato dalla totalità dell'azienda, causando problemi in fase progettuale e
                        doganale che vanno risolti a posteriori.\@
                  \item I contratti vengono revisionati su carta a mano e non sempre aggiornati nel
                        database; manca un processo e una funzione per la revisione tecnica
                        post-vendita.\@
                  \item La manutenzione è facilitata dalla mancanza di vincoli, ma il sistema è nel
                        complesso fragile e soggetto a blocchi.\@
            \end{itemize}
\end{itemize}

\end{document}