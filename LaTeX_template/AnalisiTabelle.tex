\documentclass[12pt, a4paper]{article}

% --- Preamble ---
\usepackage{amsmath}
\usepackage{amssymb}
\usepackage{graphicx}
\usepackage[margin=1in]{geometry}
\usepackage{fancyhdr}
\usepackage[colorlinks=true, allcolors=blue]{hyperref}
\usepackage[italian]{babel}
\usepackage{float}
\usepackage{cite}
\usepackage{url}
\usepackage[utf8]{inputenc}
\usepackage{booktabs}
\usepackage{hyperref}


% Document Information
% ---------------------------------
\title{Analisi delle Tabelle del Database Access}
\author{Cavina Mattia}
\date{November 5, 2025} % or specify a date, e.g., \date{September 25, 2025}

% --- Main Document ---
\begin{document}

\subsection{Ricostruire ed analizzare struttura DB}
\subsection{File Database}
L'applicativo viene gestito con tre file:
\begin{itemize}
      \item \textbf{CepiDa2024.accdb:} Front-End per le offerte di nuove commesse. Al suo interno sono presenti anche tabelle del DB.\@
      \item \textbf{INTERVDa2024.accdb:} Front-End per le richieste di assistenza, ovvero gestione ricambistica.
      \item \textbf{INTERVDa2024\_be.accdb:} Back-End con tutte le tabelle del DB.\@
\end{itemize}
\subsubsection{Analisi delle tabelle DB}
Di seguito elenco le tabelle presenti nel DB, indicando poi se sono ancora
utilizzate o meno all'interno dell'applicativo.
\begin{table}[H]
      \scalebox{0.85}{
            \begin{tabular}{lc}
                  \toprule
                  \textbf{Tabelle del Database Access} & \textbf{Attivo} \\
                  \midrule
                  Dettagli offerta                     & Sì              \\
                  \midrule
                  Dettagli offerte Optional            & Sì              \\
                  \midrule
                  Lingua\_Documenti                    & Sì              \\
                  \midrule
                  Montaggio                            & Sì              \\
                  \midrule
                  Offerte                              & Sì              \\
                  \midrule
                  Offerte\_titoli                      & Sì              \\
                  \midrule
                  Ordini                               & Sì              \\
                  \midrule
                  Agente                               & Sì              \\
                  \midrule
                  Autore offerta                       & Sì              \\
                  \midrule
                  Banche d'appoggio Cepi               & Sì              \\
                  \midrule
                  Categorie utilizzatori               & Sì              \\
                  \midrule
                  Causali                              & Sì              \\
                  \midrule
                  Clienti                              & Sì              \\
                  \midrule
                  Condizioni                           & Sì              \\
                  \midrule
                  Corrieri                             & Sì              \\
                  \midrule
                  Dettagli offerta 97                  & No              \\
                  \midrule
                  Dettagli offerta 98                  & No              \\
                  \midrule
                  Dettagli offerta 2002                & No              \\
                  \midrule
                  EX\_Prodotti\_08032022               & No              \\
                  \midrule
                  Lingue\_accessori                    & Sì              \\
                  \midrule
                  Nazioni                              & Sì              \\
                  \midrule
                  Offerte 97                           & No              \\
                  \midrule
                  Offerte 98                           & No              \\
                  \midrule
                  Offerte 2002                         & No              \\
                  \midrule
                  Pagamento                            & Sì              \\
                  \midrule
                  Prodotti                             & Sì              \\
                  \midrule
                  Riferimenti Esterni                  & Sì              \\
                  \midrule
                  Riferimenti Interni                  & Sì              \\
                  \midrule
                  Tabella di parità Euro               & Sì              \\
                  \midrule
                  tblPassword                          & Sì              \\
                  \midrule
                  Tipi di assistenza                   & Sì              \\
                  \midrule
                  Tipi di imballo                      & Sì              \\
                  \midrule
                  Tipi di percetuale Agente            & Sì              \\
                  \midrule
                  Tipi di resa                         & Sì              \\
                  \midrule
                  Utilizzatori                         & Sì              \\
                  \midrule
                  Zona                                 & Sì              \\
                  \bottomrule
            \end{tabular}}

      \caption{Tabelle del database Access esistente}\label{tab:database-tables}
\end{table}
\subsubsection{Attributi tabelle Attive}
Elenco di seguito gli attributi delle tabelle principalmente utilizzate.
\begin{itemize}
      \item\label{sec:Dettagli offerta}\textbf{Dettagli offerta} (ID offerta:OFFERTE, ID prodotto:PRODOTTI, \\
            Posizione,Prezzo unitario euro, Prezzo unitario, Quantità, Volume, Altezza, Profondità,
            Diametro, Lunghezza, Portata, Potenza installata, CosQuadro unit euro,
            CosQuadro unit, CosQuadro unit noncoll euro, CosQuadro unit noncoll, Note,
            Nuovo nome del prodotto, ID causale:CAUSALI, Comm, Data\_aggiuntiva\_no\_stampa,
            Data\_aggiuntiva\_si\_stampa, Titolo\_riga, Sub\_totale\_offerta, Tipo,
            Perc\_tipo, Visualizza\_Capitolo, Sub\_totale\_offerta\_listino)

      \item\label{sec:Offerte}\textbf{Offerte} (ID offerta, ID cliente:CLIENTI, ID autore:AUTORE OFFERTA, \ ID agente:AGENTE,
            Data Offerta, Rif utilizzatore, Spese trasporto,
            Spese montaggio, Valutata offerta, Cambio, ID euro:TABELLA DI PARITA EURO, ID pagam:PAGAMENTO, Sconto 1 offerta, Sconto 2 offerta,
            Sconto aggiuntivo, Note, Termini consegna,
            Resa, Pagamento, Imballo, Corriere, Montaggio, Garanzia, Condizioni, Volt monofase, Hz monofase, Volt trifase, Hz trifase,
            Data consegna, Validità offerta, Incidenza quadro, ID ordine:ORDINE, epilogo offerta,
            Destinatario offerta, Destinatario Fax, Schema Allegato, Schema Allegato\_2, Schema Allegato\_3,\label{sec:SchemaAllegato}
            ID tipo percentuale:TIPI PERCENTUALE AGENTE, fattura di evasione, Importo evaso, ID Banca:Banche d'appoggio cepi, Pagamento per Fatture,
            offerta\_in\_valuta, rif comm, Esclusione, Direttiva\_atex, norme\_sanitarie, Visualizza\_descr\_aggiuntiva, DB\_provenienza, Titolo, Rif\_ord\_cli,
            ValoreSconto0Agente, ValoreSconto1Agente, ValoreSconto2Agente, ValoreSconto3Agente, VerificaCapitolo, RIF\_1090, ID Tipo\_resa:TIPI DI RESA,
            Offerta\_calda, Non\_importare\_HubSpot)

      \item\label{sec:Prodotti}\textbf{Prodotti} (\underline{ID PRODOTTO}, Nome prodotto italia,Nome prodotto inglese, Nome prodotto francese, Nome prodotto tedesco,
            Nome prodotto italiaspagnolo, Nome prodotto portoghese, Nome prodotto jolly,
            Memo prodotto italia,Memo prodotto inglese, Memo prodotto francese, Memo prodotto tedesco, Memo prodotto spagnolo, Memo prodotto portoghese,
            Memo prodotto jolly, Prezzo articolo, Prezzo articolo euro, Potenza installata,
            Aria compressa, Volume imballo, Peso a vuoto, Volume, Altezza, Profondità, Diametro, Lunghezza, Portata, Costo quadro euro,
            Costo quadro, costo quadro non collegato euro, Costo qudro non collegato,
            Descrizione Manuale ita, Descrizione Manuale fra, Descrizione Manuale ing, Descrizione Manuale ted, Descrizione Manuale spa, Prezzo articolo euro ex, Tipo, perc\_tipo)

      \item \textbf{Ordini} (\underline{ID Ordine}, SubID Ordine, ID Offerta.OFFERTA, Data ordine,
            ID cliente:CLIENTI, ID pagamento:PAGAMENTO,
            ID preventivo produzione, ID spedizionieri, Destinazione:UTILIZZATORI, Destinazione Indirizzo,
            Destinazione CAP, Destinazione Città, Destinazione Prov, Destinazione Nazione, Destinazione Telefono, Destinazione Fax, Destinazione Zona,
            Destinazione2\_ragione\_soc, Destinazione2\_indirizzo, Destinazione2\_tel, Destinazione2\_fax, Commessa,
            Note ut, Etichetta, Data consegna, DataContratto, DataIscriz, DataTecnico, DataProg, DataProd, DataSpediz, DataMagaz, ID coriere:CORIERI,
            ID tipo imballo:TIPI DI IMBALLO, ID tipo\_resa,:TIPI DI RESA, Note intervento, ID Tipo\_assistenza:TIPI ASSISTENZA,
            Pervenuto\_a, Nominativo\_richiedente, Data cartella, Data evasione,  Tipo\_assistenza, N\_comm, Data fattura 1, Data fattura 2, Data fattura 3,
            Data consegna ut, Manuale\_uso\_manu, Centraline\_WP, Supervisiore, Etichette,
            Lingua\_manuale:LINGUE\_ACCESSORI, Lingua\_centraline:LINGUE\_ACCESSORI,\\
            Lingua\_supervisore:LINGUE\_ACCESSORI, Lingua\_etichette:LINGUE\_ACCESSORI,
            Data\_prev\_cons\_manuale, Data\_cons\_cliente\_manuale,
            Note\_manuale, Etichette\_cepi\_altre, Dichiaraz\_dogana, Esporta\_adhoc, Rif\_ord\_cli)

      \item \textbf{Causali} (\underline{ID causale}, Sigla, Descrizione)

      \item \textbf{Clienti} (\underline{ID cliente}, Codice\_adhoc, Contatto, Posizione, Indirrizzo, CAP, Città, Prov, Zona:ZONA, Nazione:NAZIONI, Fax,
            PIVA-CF, Banca, bancca Agenzia, banca ABI, Banca CAB, Sconto, Sconto2, Tipo valuta, perc dec trasp, perc dec mont,
            Termini consegna, Resa, Pagamento, Imballo, Coriere, Montaggio, Garanzia,
            Condizioni Aggiuntive, Monofase, Trifase, Frequenza, E-mail, Cellulare contatto, c\_ITA, c\_CEE, c\_ECEE, id riferimento interno:RIFERIMENTO INTERNO,
            id riferimento esterno:RIFERIMENTO ESTERNO, Tipo\_cliente, Logo\_cliente\_sigla, RischioCliente,
            StringaPagamentoRischio, Nome società, Ex\_nome\_SOC, Short\_name)

      \item \textbf{Autore Offerta} (\underline{ID autore}, nome autore, sigla, firma)

      \item \textbf{Agente} (\underline{ID agente}, nome agente, indirizzo, cap, città, prov, zona, tel)

      \item \textbf{Pagamento} (\underline{ID pagam}, Descrizione pagamento)

      \item \textbf{Utilizzatori} (ID utilizatori, Nome società, Contatto, Posizione, Indirizzo, CAP, Città, Prov, Zona:ZONA, Nazione:NAZIONI,
            Telefono, fax, ID CATEGORIE UTILIZZATORE, PIVA-CF)

      \item \textbf{Tipo Percentuale Agente} (\underline{ID tipo percentuale}, tipo di percentuale, Descrizione)

      \item \textbf{Banche d'appoggio cepi} (\underline{ID Banca}, Sigla, Agenzia, ABI Telex, CAB Swift Code, C/C Count, IBAN)

      \item \textbf{Tipi di resa} (\underline{ID Tipo resa}, Resa, Altro)
\end{itemize}
Dalla scrittura degli attributi si può notare che:
\begin{itemize}
      \item Non tutte le tabelle hanno chiavi assegnate.
      \item Sono presenti più istanze di una sola tipologia di attributo che potrebbero
            essere reificate in tabelle opportune, come l'attributo
            \hyperref[sec:SchemaAllegato]{Schema Allegato} nella tabella Offerte.
\end{itemize}

\subsubsection{Analisi delle relazioni Dettagli Offerta}
Si è scelto di analizzare la parte di DB relativa ai dettagli offerta
(Fig:~\ref{fig:Dettagli Offerta Sch}) in quanto è la più utilizzata e
consultata. Per ricostruirla, in assenza di cardinalità e collegamenti, si è
sfruttato lo schema ricostruito tramite la query \textit{Informazioni dettagli
      offerte}.

\begin{figure}[H]
      \centering
      \includegraphics[width=0.8\textwidth]{../image/schemi/DettagliOfferta.png}
      \caption{Schema Dettagli Offerta}\label{fig:Dettagli Offerta Sch}
\end{figure}
\subparagraph{Chiavi:}
Dallo schema scheletro ricostruito tramite la query \textit{Informazioni
      dettagli offerte} si nota come siano state eliminate le chiavi delle tabelle
Offerte e Dettagli offerte. Questi riferimenti vengono ricostruiti solo in fase
di join con le query.

\subparagraph{Cardinalità:}
Un altro dettaglio dallo schema ricostruito tramite query si nota come le
cardinalità siano state rimosse. Dall'intervista al progettista era emerso come
in una prima fase:
\begin{itemize}
      \item \textit{Prodotti} possano essere in $\infty$ \textit{Dettagli offerta}, ma un \textit{Dettagli offerta} corrisponde ad 1 \textit{Prodotti}
      \item \textit{Dettagli Offerta} si riferisce ad un solo \textit{Offerta}, ma una \textit{Offerta} può avere $\infty$ \textit{Dettagli Offerta}
\end{itemize}

\subparagraph{Normalizzazioni tabelle:}
Non essendoci chiavi di riferimento si deduce che il database non sia in prima
forma normale. Di seguito impostando un correttivo riassegnando le chiavi
originali, analizziamo se possa essere in almeno seconda forma normale,
indicando nel caso le dipendenze transitive riscontrate.

\begin{itemize}
      \item \hyperref[sec:Offerte]{Offerte:} (Key: ID OFFERTA)
            \begin{itemize}
                  \item ID\_CLIENTE $\longrightarrow$ Destinatario Offerte, Destinatario Fax.
                  \item ID\_AGENTE $\longrightarrow$ ValoreSconto0Agente, ValoreSconto1Agente,
                        ValoreSconto2Agente, ValoreSconto3Agente.
                  \item ID\_CLIENTE $\longrightarrow$ Volt, monofase, Hz monofase, Volt trifase, Hz
                        trifase.
            \end{itemize}
      \item \hyperref[sec:Prodotti]{Prodotti:} (Key: ID PRODOTTO)
            La tabella si può considerare in seconda forma normale data la chiave \textit{ID PRODOTTO}.

      \item \hyperref[sec:Dettagli offerta]{Dettagli Offerta:}  (Key: ID OFFERTA, ID PRODOTTO)
            \begin{itemize}
                  \item ID PRODOTTO $\longrightarrow$ Volume, Altezza, Profondità, Diametro, Lunghezza,
                        Portata, Potenza installata.
            \end{itemize}

\end{itemize}
I problemi fatti riscontrati in queste tabelle, come la mancanza di cardinalità, chiavi e non normalizzazione, sono comuni ad altre tabelle analizzate come \textit{Ordini};
si è voluto mostrare solo il caso più utilizzato.

\end{document}