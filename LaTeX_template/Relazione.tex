\documentclass[12pt, a4paper]{article}

% --- Preamble ---
\usepackage{amsmath}
\usepackage{amssymb}
\usepackage{graphicx}
\usepackage[margin=1in]{geometry}
\usepackage{fancyhdr}
\usepackage[colorlinks=true, allcolors=blue]{hyperref}
\usepackage[italian]{babel}
\usepackage{float}
\usepackage{url}
\usepackage{subfiles}
\usepackage{booktabs}

\pagestyle{plain}

% Document Information
% ---------------------------------
\title{Tirocinio Formativo}
\author{Cavina Mattia}
\date{\today} % or specify a date, e.g., \date{September 25, 2025}

% --- Main Document ---
\begin{document}

% --- Title Page ---
\maketitle

% --- Sections ---
\section{Introduzione}
Il progetto di tirocinio era volto all'analisi di un applicativo, su base
Microsoft Access, in uso dal reparto commerciale per gestire le vendite.
L'applicativo è in uso da circa 30 anni è stato ampliato nel tempo senza una
reale progettazione, portando a criticità e difficoltà di manutenzione. Il
progetto è stato svolto per conto e presso la sede di Cepi S.p.A.~\cite{Cepi}.
\section{Tecnologie}

\subsection{Programmi utilizzati}
\begin{itemize}
    \item Microsoft Access
    \item GitHub
    \item Visual Studio
    \item Lucidchart
\end{itemize}

\section{Attività}
Le principali attività svolte sono:
\begin{enumerate}
    \item Individuare casi d'uso.\@
    \item Intervistare utenti e progettista del DB raccogliendo eventuali criticità e
          pregi.\@
    \item Ricostruire e analizzare la struttura del DB.\@
    \item Individuare criticità e pregi dell'applicativo.\@
    \item Verificare se sono possibili operazioni correttive sullo stesso.\@
    \item Proporre eventuali nuove tecnologie applicabili.
\end{enumerate}

%File Casi d'uso
\subfile{UseCase.tex}
\newpage
%Aggiunta file con interviste
\subfile{Domande.tex}

%Aggiunta file con analisi tecnica
\subfile{AnalisiTabelle.tex}

%Conclusioni
\section{Conclusioni}
\subfile{Conclusioni.tex}

\bibliographystyle{plain}
\bibliography{references}

% --- Table of Contents ---
\tableofcontents

\end{document}